% https://raw.githubusercontent.com/matplo/alice_papers/master/twiki/alice_papers.bib
% https://twiki.cern.ch/twiki/bin/viewauth/ALICE/GuidelineEditing

\RequirePackage{fix-cm} % ???
% https://tex.stackexchange.com/search?q=svjour3
% https://tex.stackexchange.com/questions/482867/svjour3-vec-redefinition
% In order to avoid the spurious warning about size substitutions, load fix-cm

\documentclass[twocolumn,epjc3]{svjour3}

\usepackage[T1]{fontenc}
%\usepackage{mathptmx}           % no bold math fonts (in title), known bug
\usepackage{newtxtext,newtxmath} % Times fonts (text and math)

\usepackage{graphicx}
\usepackage{flushend} % !!!
%\usepackage{natbib}
%\usepackage[numbers,sort&compress]{natbib}
\usepackage{cite}
\usepackage[colorlinks,citecolor=blue,urlcolor=blue,linkcolor=blue]{hyperref}

\usepackage{xspace}
\usepackage{maybemath} % used only once (in title)
\usepackage[debugshow]{tracefnt}
\smartqed
\journalname{Eur. Phys. J. A}

\newcommand{\np}     {\ensuremath{np \rightarrow pn}\xspace}
\newcommand{\dpfrag} {\ensuremath{dp \rightarrow ppn}\xspace}
\newcommand{\dpchex} {\ensuremath{dp \rightarrow (pp)n}\xspace}
\newcommand{\dpret}  {\ensuremath{dp \rightarrow (pn)p}\xspace}
\newcommand{\GeVc}   {Ge\kern-.1emV/c\xspace}
\newcommand{\GeV}    {Ge\kern-.1emV\xspace}

%\usepackage{ragged2e}
%\usepackage{underscore}
%\usepackage{fancyhdr}


\begin{document}

\title{Charge exchange \maybebm{{\dpchex}} % correct bold math (italic) fonts
  reaction study at 1.75 A \GeVc by the STRELA spectrometer}

\author{\raggedright
  S.~N.~Basilev\thanksref{jinr}    \and Yu.~P.~Bushuev\thanksref{jinr}   \and
  S.~A.~Dolgiy\thanksref{jinr}     \and V.~V.~Glagolev\thanksref{jinr}   \and
  D.~A.~Kirillov\thanksref{jinr}   \and N.~V.~Kostyaeva\thanksref{jinr}  \and
  A.~D.~Kovalenko\thanksref{jinr}  \and A.~N.~Livanov\thanksref{jinr}    \and
  P.~K.~Manyakov\thanksref{jinr}   \and G.~Martinsk\'{a}\thanksref{upjs} \and
  J.~Musinsky\thanksref{saske}     \and N.~M.~Piskunov\thanksref{jinr}   \and
  A.~A.~Povtoreiko\thanksref{jinr} \and P.~A.~Rukoyatkin\thanksref{jinr} \and
  R.~A.~Shindin\thanksref{jinr}    \and I.~M.~Sitnik\thanksref{jinr}     \and
  V.~M.~Slepnev\thanksref{jinr}    \and I.~V.~Slepnev\thanksref{jinr}    \and
  J.~Urb\'{a}n\thanksref{upjs}
}

\thankstext{e1}{\,e-mail: contact@example.com (corresponding author)}

\institute{\noindent
  \,Joint Institute for Nuclear Research, Joliot Curie 6, 141980 Dubna,
  Moscow region, Russia\label{jinr} \and
  \,University of P.\,J. \v{S}af\'{a}rik, Jesenn\'{a} 5, 04001 Ko\v{s}ice,
  Slovak Republic\label{upjs} \and
  \,Institute of Experimental Physics, Watsonova 47, 04001 Ko\v{s}ice,
  Slovak Republic\label{saske}
}

\date{Received: date / Accepted: date}
% The correct dates will be entered by the editor
\maketitle

\begin{abstract}
  The differential cross sections of the charge exchange reaction \dpchex has
  been measured at 1.75 \GeVc per nucleon for small transferred momenta using
  the one arm magnetic spectrometer STRELA at the Nuclotron accelerator in JINR
  Dubna. The ratio of the differential cross section of the charge exchange
  reaction \dpchex to that of the \np elementary process is discussed in order
  to estimate the spin-dependent part of the \np charge exchange amplitude. The
  \np amplitude turned out to be predominantly spin-dependent.
\end{abstract}

\section{Introduction}
In the theory of nucleon-nucleon scattering extracting complex amplitudes of the
scattering matrix is a matter of fundamental importance. For all amplitudes to
be obtained, a complete experiment must be performed, \textit{i.e.}, an
experiment with a set of observed quantities providing a full and exhaustive
description of this process. Such an experiment comprises measurements with
polarized both projectile and target what is large and laborious task.

Nevertheless under certain experimental conditions, there is a possibility to
determine some amplitudes of the scattering matrix or a set of them. One of the
chances is the charge exchange reaction on the \dpchex with the use of
unpolarized protons and unpolarized deuterons, which under certain conditions is
determined only by the spin dependent amplitude of the elementary \np
scattering. When studying the differential cross section of this reaction at
small four-momentum transfer squared, it is possible to estimate the
spin-dependent term of the \np scattering amplitude in the context of the
impulse approximation. The effect can be understood qualitatively in the
following way. Two nucleons, bound in the deuteron may be in $^3S_1$ and $^3D_1$
$(T = 0)$ spatial and spin-symmetric states; their isospin is antisymmetric. In
the charge exchange at $0^\circ$ w.r.t. laboratory frame (proton rest frame),
the transition from $^3S_1$ or $^3D_1$ to a charge-symmetric $^1S_0$ or $^1D_2$
state of the two protons requires spin flip, in order to satisfy the Pauli
principle and ensure an anti-symmetric total wave function. In this way, the
spin-dependent part of the elementary charge exchange amplitude will be
reflected through the probability of the charge exchange process on the
deuteron.

The original idea to take use of the charge exchange reaction on the unpolarized
deuteron to determine the spin-dependent part of the \np charge exchange was
proposed by Pomeranchuk \cite{pom51} and Chew \cite{chew51}. Later this
possibility was emphasized in a series of works partly
\cite{mig55,pom51_2,lap57,dea72,dea72_2,ala75,ala75_2,bug87}. The mathematical
description was developed later by Dean \cite{dea72,dea72_2}. These formulas
were obtained under certain assumptions, namely relying on the validity of the
impulse and closure approximations. In the work by Lednicky and Lyuboshitz
\cite{led04} it was shown that at relativistic energies these two assumptions
are also justified.

In the general case the nucleon-nucleon ($NN$) amplitude in the centre of mass
system can be presented as \cite{gla02}
\begin{equation}
  \begin{split}
    M =\ a\ +\ &b
    (\boldsymbol{\sigma}_1\,\mathbf{n})
    (\boldsymbol{\sigma}_2\,\mathbf{n})\ +\ c\bigl[
    (\boldsymbol{\sigma}_1\,\mathbf{n}) +
    (\boldsymbol{\sigma}_2\,\mathbf{n})\bigr]\ \ + \\
    +\ &e
    (\boldsymbol{\sigma}_1\,\mathbf{m})
    (\boldsymbol{\sigma}_2\,\mathbf{m})\ +\ f
    (\boldsymbol{\sigma}_1\,\mathbf{l})
    (\boldsymbol{\sigma}_2\,\mathbf{l})\,,
  \end{split}
\end{equation}
where the orthonormal basis
\begin{equation}
  \mathbf{l} =
  \frac{\mathbf{k}_f + \mathbf{k}_i}{|\mathbf{k}_f + \mathbf{k}_i|}\,, \quad
  \mathbf{m} =
  \frac{\mathbf{k}_f - \mathbf{k}_i}{|\mathbf{k}_f - \mathbf{k}_i|}\,, \quad
  \mathbf{n} =
  \frac{\mathbf{k}_i \times \mathbf{k}_f}{|\mathbf{k}_i \times \mathbf{k}_f|}\,,
\end{equation}
introduced in \cite{gol66} is used. The unit vectors $\mathbf{k}_i$ and
$\mathbf{k}_f$ are the initial and final nucleons momenta, respectively.
$\boldsymbol{\sigma_1}$ and $\boldsymbol{\sigma}_2$ are the Pauli $2\times2$
matrices corresponding to the beam and target nucleons. The coefficients
$a, b, c, e$ and $f$ are complex scattering amplitudes which are functions of
the interacting particles energies and scattering angles.

% The vectors $\mathbf{p}_i$ and
% $\mathbf{p}_f$ are the initial and final nucleons momenta, respectively,
% $\boldsymbol{\sigma}$ and $\boldsymbol{\sigma}_i$ are the Pauli matrices of
% incident particle and $i$-th nucleon from the deuteron, respectively. The
% coefficients $a, b, c, e$ and $f$ are complex functions of the interacting
% particles energies and scattering angles.

The differential cross section of the elementary \np charge exchange can be
represented as a sum of the spin-independent (superscript $SI$) and spin-dependent
(superscript $SD$) parts
\begin{equation}
  (d\sigma/dt)_{\np} = (d\sigma/dt)^{SI}_{\np} + (d\sigma/dt)^{SD}_{\np}\,.
\end{equation}
The mathematical formalism developed in \cite{dea72, dea72_2, bug87} allows to
connect the differential cross sections of the deuteron charge exchange and the
elementary \np reactions. In the impulse approximation the $dp$ charge exchange
differential cross section at small momentum transfer $|t|$ is related to the
$NN$-amplitudes via
\begin{equation}
  \begin{split}
    (d\sigma/dt)_{\dpchex} = \bigl[1 - F_d(t)\bigr]\,(d\sigma/dt)^{SI}_{\np} \\
    + \quad
    \bigl[1 - 1/3\,F_d(t)\bigr]\,(d\sigma/dt)^{SD}_{\np}\,,
  \end{split}
\end{equation}
%where $F_d(t)$ denotes the deuteron form factor, $t$ is the 4-momentum transfer
%squared, t = (Pd - P1- P2)2, P1, P2 are the final fast
%protons four-momenta w.r.t. laboratory frame,



% \begin{figure} % one-column wide
%   \includegraphics{example.eps}
%   \caption{Please write your figure caption here}
%   \label{fig:1}
% \end{figure}

% \begin{figure*} % two-column wide
%   \includegraphics[width=0.75\textwidth]{example.eps}
%   \caption{Please write your figure caption here}
%   \label{fig:2}
% \end{figure*}

\begin{acknowledgements}
  The authors are grateful to the JINR VBLHEP directorate for supporting their
  experiment and the Nuclotron accelerator team. This research was supported by
  the Ministry of Education, Science, Research and Sport of the Slovak Republic
  (VEGA Grant No.1/0113/18).
\end{acknowledgements}

% BibTeX users please use one of
%\bibliographystyle{spbasic}      % basic style, author-year citations
%\bibliographystyle{spmpsci}      % mathematics and physical sciences
%\bibliographystyle{spphys}       % APS-like style for physics
%\providecommand{\urlprefix}{}
%\bibliography{references.bib}   % name your BibTeX data base

% bla bla (2020).
% \newblock \doi{10.1140/epja/s10050-020-00127-7}.
% \newblock \urlprefix\url{https://doi.org/10.1140/epja/s10050-020-00127-7}

\begin{thebibliography}{99}
\bibitem{pom51}
  I. Pomeranchuk, Sov. JETF \textbf{21}, 1113 (1951)
\bibitem{chew51}
  G.F. Chew, Phys. Rev. \textbf{84}, 710 (1951)
\bibitem{mig55}
  A.B. Migdal, J. Exp. Theor. Phys. (in Russian) \textbf{28}, 3 (1955)
\bibitem{pom51_2}
  I. Pomeranchuk, Dokl. Akad. Nauk (in Russian) LXXVIII, 249 (1951)
\bibitem{lap57}
  L.I. Lapidus, J. Exp. Theor. Phys. (in Russian) \textbf{32}, 1437 (1957)
\bibitem{dea72}
  N.W. Dean, Phys. Rev. D \textbf{5}, 1661 (1972)
\bibitem{dea72_2}
  N.W. Dean, Phys. Rev. D \textbf{5}, 2832 (1972)
\bibitem{ala75}
  B.S. Aladashvili et al., Nucl. Phys. B \textbf{92}, 189 (1975)
\bibitem{ala75_2}
  B.S. Aladashvili et al., Nucl. Phys. B \textbf{86}, 461 (1975)
\bibitem{bug87}
  D.V. Bugg, C. Wilkin, Nucl. Phys. A \textbf{167}, 575 (1987)
\bibitem{led04}
  R. Lednicky, V.L. Lyuboshitz, V.V. Lyuboshitz, Proc. ISHEPP XVI, 199,
  Dubna (2004)
\bibitem{gla02}
  V.V. Glagolev et al., Eur. Phys. J. A \textbf{15}, 471 (2002)
\bibitem{gol66}
  M. Goldberger, K. Watson, Collision Theory, Wiley, New York (1966)
\bibitem{gla08}
  V.V. Glagolev et al., Cent. Eur. J. Phys. \textbf{6}, 781 (2008)
\bibitem{gla13}
  V.V. Glagolev et al., Instrum. Exp. Tech. \textbf{56}, 387 (2013)
\bibitem{sha09}
  V.I. Sharov et al. Eur. Phys. J. A \textbf{39}, 267 (2009)
  % https://doi.org/10.1140/epja/i2008-10719-x
\bibitem{sha09_2}
  V.I. Sharov et al., Phys. At. Nucl. \textbf{72}, 1007 (2009) \\
  V.I. Sharov et al., Phys. At. Nucl. \textbf{72}, 1021 (2009)
  % https://doi.org/10.1134/S1063778809060131
  % https://doi.org/10.1134/S1063778809060143
\bibitem{shi11}
  R.A. Shindin et al., Phys. Part. Nucl. Lett. \textbf{8}, 90 (2011)
  % https://doi.org/10.1134/S1547477111020129
\bibitem{biz75}
  G. Bizard et al., Nucl. Phys B \textbf{85}, 14 (1975)
\bibitem{bys78}
  J. Bystricky, F. Lehar, Nucleon-Nucleon Scattering data, Karlsruhe:
  Fachinformationszentrum, 521, (1978)
\bibitem{tro14}
  Yu.A. Troyan et al., Phys. Part. Nucl. Lett. \textbf{11}, 101 (2014)
\bibitem{bas14}
  S.N. Basilev et al., PoS(Baldin ISHEPP XXII), 137, 2014
\bibitem{bas16}
  S.N. Basilev et al., J. Phys. Conf. Ser. \textbf{678}, 012040, 2016
\end{thebibliography}

\end{document}

%%% Local Variables:
%%% mode: latex
%%% TeX-master: t
%%% End:
